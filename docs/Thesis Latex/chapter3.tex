\chapter{مروری بر کارهای گذشته}
از آنجا که سیستم‌های پرسش و پاسخ در تمامی زبان‌ها از اهمیت بالایی برخوردار هستند تحقیقات و پروژه‌های بسیاری با روش‌های متنوع برای پاسخ‌دهی به این مسئله ارائه شده است. در ادامه به بررسی تعدادی از روش‌های پیاده‌سازی شده که سعی در ارائه‌ی پاسخ و یا بهبود اینگونه سیستم‌ها را داشته‌اند می‌پردازیم. برخی از این روش‌ها برای پاسخ‌دهی به سوالات ساده و برخی برای سوالات پیچیده طراحی شده‌اند و همگی آنها مبتنی بر پایگاه دانش می‌باشند.

\section{روش‌های مبتنی بر ارائه‌ی معنایی}
در این روش‌ها تلاش می‌شود تا از سوال ورودی بازنمایی‌ای\footnote{\lr{Representation}} استخراج شود که حاوی معنای نهفته در سوال باشد تا بر اساس آن بتوان به کوئری متناسب با سوال رسید. در یکی از این روش‌ها از تجزیه‌کننده‌های\footnote{\lr{Parser}} موجود برای زبان انگلیسی استفاده می‌شود. هدف تولید یک کوئری الگو\footnote{\lr{Template}} متناسب با بازنمایی معنایی\footnote{\lr{Semantic}} تولید شده می‌باشد. منظور از کوئری الگو یک کوئری \lr{SPARQL} ناقص است که قالب آن کامل بوده اما بدون شناسه می‌باشد. یک کوئری الگو دارای جایگاه‌های مشخص برای قرار گرفتن شناسه‌های مرتبط با سوال می‌باشد. در این روش در تجزیه‌کننده‌ی مورد استفاده تغییراتی ایجاد شده که در حین تجزیه‌ی جمله بتواند کوئری الگو متناسب را تولید نماید. در مرحله‌ی بعدی موجودیت‌های جمله تشخیص داده شده و در جایگاه‌های مربوطه قرار می‌گیرند تا کوئری‌های نهایی تولید شوند. با امتیازدهی این کوئری‌ها برترین آنها انتخاب شده و استفاده می‌شود \cite{unger2012tembased}.
\\در روشی دیگر بدون استفاده از تجزیه‌کننده‌های موجود یک بازنمایی معنایی تولید می‌شود. بدین منظور برخلاف روش قبلی، ابتدا موجودیت‌های جمله تشخیص داده شده و استخراج می‌شوند. سپس به روش تکرارشونده\footnote{\lr{Iterative}} ارائه‌ی مورد نظر به صورت یک گراف تولید می‌شود. در هر تکرار این مرحله با توجه به گراف و اجزای جمله و موجودیت‌های تشخیص داده شده تعدادی عمل قابل انجام است که بوسیله‌ی آنها گره‌هایی به گراف اضافه می‌شوند. در نهایت با استفاده از این گراف یک کوئری تولید می‌شود \cite{sorokin2017endtoendweak}.
\section{روش‌های مبتنی بر تشخیص رابطه}
همانطور که در فصل پیشین توضیح داده شد، گراف‌های دانش مجموعه‌ای از موجودیت‌ها هستند که توسط روابطی بهم متصل شده‌اند. در بسیاری از سیستم‌های پرسش و پاسخ هدف سیستم تشخیص موجودیت و رابطه‌ی مورد نظر سوال می‌باشد تا بوسیله‌ی آن بتوان با ایجاد کوئری موجودیت پاسخ را دریافت نمود. در یکی از روش‌های موجود کل فرایند به چهار قسمت شامل تشخیص موجودیت، پیوند دادن موجودیت، تشخیص رابطه و دریافت پاسخ تقسیم شده است که قسمت تشخیص رابطه به صورت یک دسته‌بند عمل می‌کند. در چنین روشی هر کدام از مراحل مستقل از بقیه‌ی مراحل می‌توانند بررسی شوند. هر کدام از سه مرحله‌ی اول توسط روش‌های مبتنی بر شبکه عصبی و یا غیر شبکه عصبی مورد آزمایش قرار گرفته‌اند و نتایج نشان از عملکرد نزدیک این دو دسته روش‌ها در این مسائل می‌دهد. در مرحله‌ی آخر این روش، از روابط و موجودیت‌های کاندید استفاده می‌شود تا موجودیت پاسخ مورد نظر یافت شود. این روش تنها توانایی پاسخ دادن به سوالات ساده را دارد \cite{mohammed2018strongbaseline}.\\
در روشی مشابه نیز ابتدا روابط و موجودیت‌های کاندید استخراج می‌شوند، اما برای توانایی پاسخ دادن به سوالات پیچیده از همگی این روابط و موجودیت‌ها هم‌زمان استفاده می‌شود. بدین منظور این روش تلاش می‌کند مسیری در گراف دانشی که استفاده می‌کند پیدا کند که همگی روابط و موجودیت‌های استخراج شده از سوال در آن مسیر وجود داشته باشد. چنین مسیری گره‌های دیگری را نیز دربر می‌گیرد که همان پاسخ‌های کاندید برای سوال می‌باشند \cite{zafar2018formalquery}.\\
یکی از ایراد‌های وارد بر دو روش فوق در این است که به دلیل مستقل بودن مراحل تشخیص موجودیت و تشخیص رابطه امکان تشخیص اشتباه بالا می‌رود. به عنوان مثال برای کلماتی که چندین موجودیت مشابه به آن‌ها در گراف دانش یافت می‌شود، با توجه به رابطه‌ی مورد پرسش می‌توان موجودیت مناسب‌تر را تشخیص داد. به همین دلیل در روشی دیگر روابط، موجودیت‌ها و سایر گره‌های گراف دانش همگی تحت عنوان یک عنصر در نظر گرفته شده و سعی می‌شود عناصر مرتبط با پرسش تولید شده و یک مسیر صحیح در گراف دانش متناسب با پرسش مورد نظر یافت شود \cite{lan2019kbtopic}.
\section{روش‌های تولیدکننده‌ی جمله‌ی پاسخ}
در روش‌های معرفی شده در دو قسمت قبلی خروجی نهایی مدل‌ها یک موجودیت از پایگاه دانش است که به عنوان پاسخ به پرسش ورودی برگردانده می‌شود. اما در برخی روش‌ها به این پاسخ قناعت نشده و تلاش می‌شود یک جمله‌ی کامل به عنوان پاسخ بازگردانده شود. به عنوان مثال در پاسخ به پرسش "پایتخت ایران کجاست" به جای کلمه یا موجودیت "تهران"، جمله‌ی "پایتخت ایران تهران است" نمایش داده می‌شود.
برای پیاده‌سازی چنین سیستمی، در \cite{yini2016neuralgenerative} از یک شبکه‌ی دنباله به دنباله استفاده می‌شود. همانطور که مشخص است برخی از کلمات خروجی از پایگاه دانش استخراج می‌شوند و این شبکه نمی‌تواند به تنهایی تمامی جمله‌ی پاسخ را تولید نماید. کلمات خروجی را می‌توان به دو دسته‌ تقسیم نمود؛ کلمات پایگاه که از پایگاه دانش استخراج می‌شوند و کلمات مشترک که برای تکمیل جمله تولید می‌شوند و نیازی به پایگاه برای تولید آنها نمی‌باشد. برای استفاده هم‌زمان این شبکه و پایگاه دانش، تغییراتی در قسمت کدگشای شبکه‌ی دنباله به دنباله داده شده است. در این قسمت در هر گام علاوه بر کلمه‌ی خروجی یک احتمال نیز تولید می‌شود که نشان می‌دهد آیا این کلمه از کلمات مشترک یا از کلمات پایگاه می‌باشد. در صورتی که تشخیص داده شود که در این گام یک کلمه‌ی پایگاه بایستی تولید شود، از پایگاه دانش استفاده شده و بوسیله‌ی موجودیت‌های یافت شده در جمله‌ی ورودی موجودیت مربوطه به جمله‌ی خروجی استخراج می‌گردد.
\section{جمع‌بندی}
همانطور که در این فصل توضیح داده شد یکی از رویکردهای متداول در این حوزه استفاده از دسته‌بندها برای تشخیص رابطه می‌باشد. در این پژوهش نیز سیستمی با این رویکرد پیشنهاد داده می‌شود که در ادامه به معرفی آن می‌پردازیم.