\chapter{مقدمه}
\section{مقدمه}
هوش مصنوعی یا هوش ماشینی به هوشمندی سیستم‌هایی گفته می‌شود که می‌توانند عملکردی مشابه با رفتارهای هوشمندانه‌ی انسان داشته باشند. این عبارت که در مقابل هوش طبیعی موجود در انسان بوجود آمده توسط آقای مک‌کارتی\footnote{\lr{John McCarthy}}، استاد دانشگاه دارتموث\footnote{\lr{Dartmouth College}} در سال 1955 بیان گردیده است. رفتارهای هوش مصنوعی شامل درک شرایط، شبیه‌سازی فرایند تفکر، توانایی استدلال و یادگیری و کسب دانش می‌شود.
ساختار مغزی انسان به‌قدری پیچیده است که چندین حوزه‌ی تخصصی برای بررسی، آموزش و تحلیل مسائل مرتبط با آن بوجود آمده است. به‌ جرأت می‌توان گفت که الفاظی نظیر یادگیری، آموزش و تقلید (که همگی بر یادگیری تاکید دارند) از جذاب‌ترین حوزه‌هایی هستند که بشر به‌مطالعه آنها پرداخته و از عجایب و شگفتی‌‌های آن پرده بر‌می‌دارد. در این میان،‌ دسته‌بندی‌هایی جزئی‌تر برای بررسی دقیق‌تر هر‌یک از مسائل مرتبط با موارد فوق ‌پایه‌گذاری شده‌اند. این دسته‌بندی شامل سه‌حوزه‌ی بینایی، پردازش زبان و مطالعه‌ی سیستم گفتاری انسان است. اگر موجودی شبیه‌ انسان است،‌ باید این سه‌ویژگی را به بهترین شکل در ساختار وجودی خود به‌همراه داشته باشد. این سه ویژگی، هرچند انسان کامل (از لحاظ درک) را نتیجه نخواهند داد، اما تا حد مطلوبی یک تقریب از انسان را نتیجه می‌دهد. علاوه بر این  با دست‌یافتن به ساختار‌هایی مجهز به عملکرد شبیه به مغز انسان، می‌توان محصولات جدیدی را خلق کرد که بخش زیادی از نیازهای انسان را پاسخ‌گو خواهند بود. دست‌یابی به چنین ساختار‌هایی اگرچه جذاب، بسیار دشوار می‌باشد. یکی از اصلی‌ترین چالش‌های موجود بحث درک\footnote{\lr{Perception}} و فهماندن یا آموزش آن به ساختار‌های مجهز به تکنولوژی هوش مصنوعی است. درک ابعاد مختلفی دارد و از آن میان می‌توان به درک تصویر، درک زبان طبیعی و درک گفتار اشاره کرد.
آنچه که در این پروژه مورد بررسی قرار می‌گیرد در زیر دسته‌ی اصلی پردازش زبان طبیعی قرار دارد. در ادامه به تعریف مسئله ‌پرداخته می‌شود.
\section{تعریف مسئله}
جستجوی اطلاعات یکی از فعالیت‌های اصلی کاربران در فضای وب است و وقتی اطلاعات مد نظر خاص و کوتاه می‌شود موتورهای جستجو نمی‌توانند پاسخگوی نیاز کاربران باشند. در این شرایط سیستم‌های پرسش و پاسخ\footnote{\lr{Question Answering System}} به بهترین نحو می‌توانند نیاز کاربر را برآورده کنند.
سیستم‌های پرسش و پاسخ در دو نوع دسته‌بندی می‌شوند. نوع اول سیستم‌هایی هستند که از یک متن پاسخ سوال را استخراج می‌کنند و نوع دوم سیستم‌هایی هستند که از یک گراف دانش\footnote{\lr{Knowledge Graph}} به عنوان پایگاه داده استفاده می‌کنند و پاسخ را از این گراف استخراج می‌کنند. محصول مد نظر این پروژه از نوع دوم بوده و گراف دانش فارسی فارس‌بیس اساس کار آن می‌باشد. 
گراف دانش یک پایگاه داده است که اطلاعات را به صورت یک گراف بیان می‌کند. در این گراف هر یک از گره‌ها بیانگر یک موجودیت\footnote{\lr{Entity}} است. بین هر دو موجودیت ممکن است رابطه‌ای وجود داشته باشد که این رابطه بوسیله‌ی اتصال یک یال بین دو گره‌ی مربوطه بیان می‌شود.
سوالات گراف دانش به دو دسته‌ی سوالات ساده و پیچیده تقسیم می‌شوند. سوالات ساده سوال‌هایی هستند که برای پاسخ‌دهی به آنها تنها نیاز به یک رابطه است ولی سوالات پیچیده نیاز به استنتاج و ترکیب چند رابطه دارند.
هدف از این پروژه طراحی سامانه‌ای است که کاربر بتواند بوسیله‌ی پرسش سوال اطلاعات مد نظر خود را کسب کند. این سیستم محدود به سوالات ساده‌ی فارسی می‌باشد و سوالات پیچیده در آن پشتیبانی نمی‌شود.

\section{تعریف زیرمسئله‌ها}
روش کار این سامانه به این صورت است که ابتدا از سوال ورودی اطلاعات مورد نیاز را استخراج می‌کند، سپس بوسیله‌ی این اطلاعات کوئری مورد نظر برای درخواست از گراف دانش را تولید کرده و پاسخ را از گراف دریافت برای کاربر نمایش می‌دهد. هسته این سامانه دارای سه بخش زیر می‌باشد:
\begin{itemize}
	\item تشخیص رابطه\footnote{\lr{Relation Classification}}
	\item تشخیص موجودیت‌ها\footnote{\lr{Entity Recognation}}
	\item تولید کوئری\footnote{\lr{Query Generation}}
\end{itemize}

\subsection{تشخیص رابطه}
این سامانه از دسته‌های "پایتخت یک کشور", "کارگردان یک فیلم", "همسر یک شخص" و ۴۵ دسته‌ی دیگر پشتیبانی می‌کند. هر کدام از این دسته‌ها یک نگاشت از یک نوع رابطه در گراف دانش مورد استفاده می‌باشد.
به عنوان مثال پرسش "پایتخت ایران کجاست؟" از دسته‌ی "پایتخت یک کشور" می‌باشد که نماینده‌ی رابطه‌ی "\lr{capital}" در گراف دانش مورد استفاده می‌باشد.
وظیفه‌ی اول سامانه تشخیص رابطه‌ی مورد پرسش در سؤال است که توسط یک دسته‌بند\footnote{\lr{Classifier}} قابل انجام می‌باشد. برای انجام این بخش از دو دسته‌بند ماشین بردار پشتیبان\footnote{\lr{Support Vector Machine}} و شبکه‌ی عصبی پیچشی\footnote{\lr{Convolutional Neural Network}} استفاده شده و هر کدام به صورت جداگانه مورد آزمایش قرار گرفته شده‌اند.

\subsection{تشخیص موجودیت‌ها}
وظیفه‌ی دوم سامانه استخراج موجودیت مورد پرسش سؤال می‌باشد. برای این منظور بایستی که تمامی موجودیت‌های نام‌دار سؤال تشخیص داده شوند. منظور از موجودیت نام‌دار انواع اسامی از جمله اشخاص، آثار، سازمان‌ها و مکان‌ها می‌باشد. به عنوان مثال جمله‌ی روبرو را در نظر بگیرید:
"علی دایی متولد اردبیل است و سابقه‌ی بازی در پرسپولیس را دارد."
مجموعه کلمات "علی دایی"، "اردبیل" و "پرسپولیس" موجودیت‌های نام‌دار این جمله می‌باشند. \\
با تشخیص موجودیت‌های نام‌دار جمله‌ی سؤال می‌توان موجودیت مورد پرسش و گره‌ی مربوط به آن در گراف دانش را استخراج نمود. به عنوان مثال در جمله‌ی "پایتخت ایران کجاست؟" کلمه‌ی "ایران" تنها موجودیت نام‌دار جمله است که همان موجودیت مورد پرسش سؤال می‌باشد.

\subsection{تولید کوئری}
سامانه در مرحله‌ی سوم از رابطه‌ و موجودیت استخراج شده از سؤال استفاده و کوئری \lr{SPARQL} مناسب را تولید کرده و بوسیله‌ی آن پاسخ سؤال را از گراف دانش استخراج می‌نماید.
\section{خلاصه فصل‌های بعد}
در فصل بعدی با مفاهیم اولیه هوش‌ مصنوعی و مدل‌های یادگیری ماشین مرتبط آشنا خواهیم شد. در ادامه تعدادی از مطالعات گذشته‌ مرور شده و به توضیح سیستم پیشنهادی این پروژه می‌پردازیم. در نهایت قسمت‌های مختلف سیستم مورد ارزیابی قرار گرفته و پیشنهادات برای بهبود آن مطرح می‌شوند.