\chapter{جمع‌بندي و نتيجه‌گيري و پیشنهادات}
%%%%%%%%%%%%%%%%%%%%%%%%%%%%%%%%%%%%%%%%%%%

\section{جمع‌بندی و نتیجه‌گیری}
در این پروژه به طراحی و پیاده‌سازی یک سیستم پرسش و پاسخ خودکار برای زبان فارسی پرداختیم. این سیستم بر اساس یک پایگاه دانش پیاده‌سازی شده و توانایی پاسخ دادن به سوالات ساده را دارا می‌باشد. هدف سیستم‌های پرسش و پاسخ وابسته به گراف دانش\footnote{\lr{Question Answering over Knowledge Graph (QA-KG)}} امکان دسترسی راحت‌تر کاربران به اطلاعات گراف دانش می‌باشد به طوری که کاربر نیاز به دانستن ساختار موجود در گراف را نداشته باشد \cite{huang2019KGembedding}. \\
سیستم موجود در این پروژه را به سه زیرمسئله‌ی ۱. تشخیص رابطه، ۲. تشخیص موجودیت و ۳. تولید کوئری تقسیم کرده و هر کدام را تحلیل و پیاده‌سازی کردیم. در نهایت با ارزیابی هر کدام از قسمت‌های سیستم توانستیم آنرا با یک سیستم مشابه مقایسه کنیم. این پروژه توانست بهبود و عملکرد مناسبی را در چنین سیستم‌هایی بدست بیاورد و همانطور که در فصل قبل ملاحظه شد، این سیستم طی آزمایش‌های مختلف توانست صحت بالای ۹۵\lr{\%} و در بهترین حالت صحت ۷۲.۹۹\lr{\%} را در تشخیص روابط بدست بیاورد. همچنین این سیستم در تشخیص موجودیت‌ها عملکرد خوبی داشته و توانست امتیاز $F_{1}$ ۳۳.۸۹\lr{\%} را بدست بیاورد. مشکل اصلی سیستم همانطور که در قسمت ارزیابی یکپارچه‌ی سیستم در فصل قبل بررسی شد، در نگاشت گره‌ی نادرست از گراف دانش به موجودیت‌های شناخته شده است که بهبود این قسمت وابسته به پایگاه دانش مورد استفاده می‌باشد.
\section{کارهای آینده}
از قدم‌های آتی برای بهبود و پیشرفت این سیستم می‌توان به موارد زیر اشاره کرد: 
\begin{itemize}
	\item ریشه‌یابی کلمات و استفاده از ریشه‌ی کلمات برای تشخیص رابطه و موجودیت. این پیش‌پردازش از بروز مشکلاتی نظیر آنچه در ارزیابی دسته‌بند شبکه عصبی پیچشی مشاهده شد جلوگیری می‌کند.
	\item در مرحله‌ی شناسایی رابطه‌، ترتیب کلمات موجود در پرسش اهمیت دارند. به عنوان مثال دو جمله‌ی "پایتخت کشور \lr{x} کدام است" و "\lr{x} پایتخت کدام کشور است" دارای کلمات یکسانی می‌باشند اما ترتیب آنها باعث می‌شود که جهت رابطه برعکس شود. با استفاده از یک شبکه‌ی عصبی بازگشتی طی مرحله‌ی تشخیص رابطه، می‌توان ترتیب کلمات را در نظر گرفته و جهت روابط را نیز تشخیص داد.
	\item
	 پیاده‌سازی دسته‌بند چندکلاسه برای تشخیص هم‌زمان چند رابطه از یک پرسش از نوع ساده.  به عنوان مثال پرسش "پایتخت و زبان رسمی کشور ایران چیست" در واقع دو سوال ساده در قالب یک سوال می‌باشد. این کار نیاز به جمع‌آوری حجم زیادی از داده برای آموزش دارد.
	\item بهبود عملکرد پایگاه دانش در تشخیص گره‌ی مربوطه به موجودیت‌ها.
	\item
ایجاد رابط برنامه‌نویسی کاربردی جداگانه‌ای در پایگاه دانش برای سوالاتی که موجودیت سوال از نوع عدد یا آدرس اینترنتی می‌باشد.
	\item
	 اضافه کردن قابلیت پاسخ‌دهی به سوالات پیچیده. در حال حاضر به این منظور روش‌هایی نظیر
	 \cite{zafar2018formalquery} و \cite{unger2012tembased}
	 ارائه شده‌اند.
\end{itemize}