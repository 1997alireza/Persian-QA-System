%% -!TEX root = AUTthesis.tex
% در این فایل، عنوان پایان‌نامه، مشخصات خود، متن تقدیمی‌، ستایش، سپاس‌گزاری و چکیده پایان‌نامه را به فارسی، وارد کنید.
% توجه داشته باشید که جدول حاوی مشخصات پروژه/پایان‌نامه/رساله و همچنین، مشخصات داخل آن، به طور خودکار، درج می‌شود.
%%%%%%%%%%%%%%%%%%%%%%%%%%%%%%%%%%%%
% دانشکده، آموزشکده و یا پژوهشکده  خود را وارد کنید

\faculty{دانشکده مهندسی کامپیوتر}
% گرایش و گروه آموزشی خود را وارد کنید
\department{گرایش نرم‌افزار}
% عنوان پایان‌نامه را وارد کنید
\fatitle{طراحی و پياده‌سازی سيستم پرسش و پاسخ خودکار سوالات ساده فارسی
\\[.75 cm]
}
% نام استاد(ان) راهنما را وارد کنید
\firstsupervisor{دکتر سعیده ممتازی}
%\secondsupervisor{استاد راهنمای دوم}
% نام استاد(دان) مشاور را وارد کنید. چنانچه استاد مشاور ندارید، دستور پایین را غیرفعال کنید.
%\firstadvisor{دکتر امیر کلباسی}
%\secondadvisor{استاد مشاور دوم}
% نام نویسنده را وارد کنید
\name{علیرضا }
% نام خانوادگی نویسنده را وارد کنید
\surname{ترابیان}
%%%%%%%%%%%%%%%%%%%%%%%%%%%%%%%%%%
\thesisdate{شهریور ۱۳۹۹}

% چکیده پایان‌نامه را وارد کنید
\fa-abstract{امروزه با در دسترس قرار گرفتن انبوهی از اطلاعات یکی از مشکلات پیش رو چگونگی جستجو و یافتن اطلاعات مد نظر در میان این انبوه داده می‌باشد. از راحت‌ترین راه‌های جستجو برای یک کاربر عادی پرسش اطلاعات مد نظر خود است. به همین منظور سیستم‌های پرسش و پاسخ یکی از مهم‌ترین کاربردهای موجود در جستجوی اطلاعات می‌باشند. 
	هدف از این پروژه طراحی سامانه‌ای است که در آن امکان پرسش سوالات ساده فارسی فراهم شده باشد. این سوالات می‌توانند در دسته‌بندی‌های مختلف از جمله پایتخت کشور، کارگردان فیلم یا سریال و ... باشد و پاسخ آنها با استفاده از گراف دانش فارسی فراهم می‌گردد. این پروژه مبتنی بر زبان برنامه‌نویسی پایتون پیاده‌سازی و ارزیابی می‌شود. مدل‌های پیاده‌سازی شده در این پژوهش بر اساس روش‌های ماشین بردار پشتیبان و شبکه‌ی عصبی پیچشی می‌باشد. هر دو روش مذکور در آزمون‌های مختلف بررسی و خطاهای هر کدام مورد بررسی قرار گرفته شده است. تمامی آزمایش‌های این پروژه به صورت با نظارت انجام گرفته است. علاوه بر موارد فوق اثر تغییر مدل‌های مختلف برای نگاشت جملات به بردار ویژگی نیز بررسی شده است. نتایج آزمایش‌ها بر روی دادگانی که در راستای همین پروژه تهیه شده است نشان می‌دهد که دقت روش پیشنهادی در پاسخ‌گویی به سوالات فارسی با استفاده از اطلاعات گراف دانش فارسی ۰۷.۲۹ درصد می‌باشد.
}
%توجه: ‌در اعداد اعشاری قسمت صحیح و اعشار جا به جا باید نوشته شوند تا در متن به طور صحیح نمایش داده شوند. به عنوان مثال در متن بالا عدد ۲۹.۰۷ را جا به جا وارد کرده تا در متن به طور صحیح نمایش داده شود. 

% کلمات کلیدی پایان‌نامه را وارد کنید
\keywords{سیستم پرسش و پاسخ، شبکه‌ی عصبی پیچشی، ماشین بردار پشتیبان، گراف دانش، سوال ساده‌ی فارسی}


\AUTtitle
%%%%%%%%%%%%%%%%%%%%%%%%%%%%%%%%%%
\vspace*{7cm}
\thispagestyle{empty}
\begin{center}
\includegraphics[height=5cm,width=12cm]{besm}
\end{center}